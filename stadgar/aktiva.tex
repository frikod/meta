\documentclass{article}
\usepackage[utf8]{inputenc}

\title{STADGAR FÖR FÖRENINGEN FRIKOD}
\author{ }
\date{ }

\begin{document}

\maketitle

\section{§ Föreningens namn och säte}
Föreningens namn är Frikod.\\
Föreningen har sitt säte i Lund.

\section{§ Föreningens ändamål}
Föreningen har till ändamål att utveckla allmännyttig programvara och dela dessa program under öppen källkod licenser.\\
Föreningen är ideell samt partipolitiskt och religiöst obunden.\\
Föreningen är demokratisk.\\
Föreningen skall uppfylla sitt ändamål genom att dess medlemmar skall tillsammans eller individuellt utveckla programvara och att programmet sedan skall få en plattform att publiceras och delas.

\section{§ Medlemskap}
Föreningen är öppen för den som uppfyller samtliga av följande villkor:
\begin{enumerate}
  \item Man har kompetens inom programvaruutveckling eller annan kompetens som kan hjälpa föreningen att driva dess ändamål vidare.
  \item Man har förmågan att godkänna, samt godkänner föreningens stadgar och att man skall förhålla sig till dessa.
  \item Man är en fysisk person
  \item Man betalar medlemsavgiften till föreningen.
\end{enumerate}
Medlemmar har rätt att bidra med programvarukod till föreningens projekt och vid ett sådant fall erkänner medlemmen att deras bidrag är del av och skyddas av källkods licensen för det projektet medlemmen bidragit till.\\
Medlemsavgift fastställs av årsmötet.\\
Medlemskapet i föreningen ska förnyas aktivt av medlemmen varje år. Medlemskapet gäller för det innevarande år som medlemmen går med eller förnyelse sker.\\
Medlemskap ska uteslutas om en medlem inte följer stadgarna eller kränker andan av vilket föreningens ändamål är. Detta sker genom beslut av årsmötet, om $\frac{2}{3}$ (två tredjedelars) majoritet röstar för att medlemmens agerande motarbetat föreningens ändamål.  \\
Utträde sker genom att skriftligt eller via e-post meddela styrelsen om ens önskan att utträda från föreningen.
\section{§ Årsmöte}
Föreningens högsta beslutande organ är årsmötet\\
Årsmötet skall hållas en gång varje kalenderår.\\
Kallelse till årsmötet sker elektroniskt eller skriftligt och ska skickas ut två veckor innan mötet.\\
Övriga ärenden som medlemmar önskar ta upp vid årsmötet skall vara till styrelsen tillhanda senast en vecka innan mötet.
\section{§ Ärenden vid årsmötet}
Vid årsmötet ska följande ärenden behandlas:
\begin{enumerate}
  \item Årsmötets öppnande.
  \item Fastställande av röstlängd.
  \item Val av mötesordförande och sekreterare.
  \item Val av en person som jämte ordföranden justerar mötets protokoll.
  \item Godkännande att kallelse skett i enlighet med stadgarna.
  \item Fastställande av dagordningen.
  \item Föredragning av styrelsens verksamhetsberättelse.
  \item Fråga om ansvarsfrihet för styrelsen för föregående verksamhetsår.
  \item Fastställande av medlemsavgifter.
  \item Val av ordförande, tillika ledamot i styrelsen.
  \item Val av övriga ledamöter.
  \item Val av valberedare jämte suppleant/er.
  \item Övriga frågor.
  \item Årsmötets avslutande.
\end{enumerate}
\section{§ Extra årsmöte}
Extra årsmöte ska hållas när minst $\frac{2}{3}$ (två tredjedelar) av medlemmarna skriftligen begär det eller när styrelsen anser det nödvändigt.
\section{§ Styrelsen}
Styrelsen ska bestå av:
\begin{itemize}
    \item 2 ledamöter
\end{itemize}
Årsmötet kan även välja en revisor om årsmötet tycker detta är lämpligt för föreningen.\\
Mandattiden för ledamöterna är 1 år.\\
Beslut om att ändra styrelsens storlek fattas av årsmötet.\\
Styrelsen är beslutsför när mer än hälften av ledamöterna är närvarande och samtliga ledamöter kallats och underrättats om sammanträdet.\\
Styrelsebeslut kräver minst $\frac{2}{3}$ (två tredjedelar) av de närvarandes röster.\\
Firmatecknare utses av styrelsen.
\section{§ Styrelsens åligganden m.m}
Styrelsen är, när årsmötet inte är samlat, föreningens högst beslutande organ.
Det åligger styrelsen:
\begin{enumerate}
  \item Att verkställa av årsmötet fattade beslut.
  \item Att handha föreningens ekonomiska angelägenheter och låta föra räkenskaper
\end{enumerate}
\section{§ Verksamhetsår}
Föreningens verksamhetsår följer kalenderåret.
\section{§ Stadgeändring}
Årsmötet kan besluta om tillägg och ändring av stadgar.\\
Giltigt beslut kräver majoritetsröst om minst $\frac{2}{3}$ (två tredjedelar) av närvarande medlemmar.
\newpage
\section{§ Föreningens upplösning}
Beslut om upplösning av föreningen kan ske om minst $\frac{3}{4}$ (tre fjärdedelars) majoritet vid årsmötet två stycken årsmöten på rak, alternativt genom ett enhälligt ordinare årsmöte. Föreningens kvarvarande tillgångar ska gå till/användas för den fortsatta positiva påverkan av den öppna källkod gemenskapen.\\\\

Antagna vid årsmötet den 15 Januari 2023.



\end{document}
